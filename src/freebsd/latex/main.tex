% !TEX program = xelatex

\documentclass[notitlepage,lecture,print]{report}

\usepackage[UTF8]{ctex}
\usepackage{graphicx}

%%%%%%%%%%%%%%%%%%%%%%%%%%%%%%%%%%%%%%%%%%%%%%%%%%%%%%%%%%%%%%%%%%%
% 西文字体:Monaco
% 中文字体:文泉驿微米黑
%%%%%%%%%%%%%%%%%%%%%%%%%%%%%%%%%%%%%%%%%%%%%%%%%%%%%%%%%%%%%%%%%%%
\usepackage{fontspec}
\setmainfont{Monaco}
\setsansfont{Monaco}
\setmonofont{Monaco}
\setCJKmainfont{WenQuanYi Micro Hei}

%%%%%%%%%%%%%%%%%%%%%%%%%%%%%%%%%%%%%%%%%%%%%%%%%%%%%%%%%%%%%%%%%%%
% 添加顔色:solarized light
%%%%%%%%%%%%%%%%%%%%%%%%%%%%%%%%%%%%%%%%%%%%%%%%%%%%%%%%%%%%%%%%%%%
\usepackage{xcolor-solarized}
\usepackage{xcolor}

%%%%%%%%%%%%%%%%%%%%%%%%%%%%%%%%%%%%%%%%%%%%%%%%%%%%%%%%%%%%%%%%%%%
% 页边距
%%%%%%%%%%%%%%%%%%%%%%%%%%%%%%%%%%%%%%%%%%%%%%%%%%%%%%%%%%%%%%%%%%%
\usepackage{geometry}
\geometry{top=15mm,bottom=15mm,left=15mm,right=15mm}

%%%%%%%%%%%%%%%%%%%%%%%%%%%%%%%%%%%%%%%%%%%%%%%%%%%%%%%%%%%%%%%%%%%
% 页眉页脚
%%%%%%%%%%%%%%%%%%%%%%%%%%%%%%%%%%%%%%%%%%%%%%%%%%%%%%%%%%%%%%%%%%%
\usepackage{titleps}
\newpagestyle{main}[\small\bfseries]{
    \sethead
    [][\href{https://github.com/XiMen-classroom}{\color{gray} \large \texttt{XiMen-classroom}}][] % 偶数页
    {}{\href{https://github.com/XiMen-classroom}{\color{gray} \large \texttt{XiMen-classroom}}}{} % 奇数页
    \headrule % 页眉画线
    \setfoot
    {}{-\quad\thepage\quad-}{} % 页脚奇偶页相同
    \footrule % 页脚画线
}
\pagestyle{main}

%%%%%%%%%%%%%%%%%%%%%%%%%%%%%%%%%%%%%%%%%%%%%%%%%%%%%%%%%%%%%%%%%%%
% 脚注样式
%%%%%%%%%%%%%%%%%%%%%%%%%%%%%%%%%%%%%%%%%%%%%%%%%%%%%%%%%%%%%%%%%%%
\usepackage{pifont}
\renewcommand{\thefootnote}{\ding{\numexpr171+\value{footnote}}}
\renewcommand{\thempfootnote}{\fnsymbol{mpfootnote}}

%%%%%%%%%%%%%%%%%%%%%%%%%%%%%%%%%%%%%%%%%%%%%%%%%%%%%%%%%%%%%%%%%%%
% 附录
%%%%%%%%%%%%%%%%%%%%%%%%%%%%%%%%%%%%%%%%%%%%%%%%%%%%%%%%%%%%%%%%%%%
\usepackage[toc,page]{appendix}

%%%%%%%%%%%%%%%%%%%%%%%%%%%%%%%%%%%%%%%%%%%%%%%%%%%%%%%%%%%%%%%%%%%
% 缩小列表环境的行间距
%%%%%%%%%%%%%%%%%%%%%%%%%%%%%%%%%%%%%%%%%%%%%%%%%%%%%%%%%%%%%%%%%%%
\usepackage{enumitem}
\setlist{nolistsep}

%%%%%%%%%%%%%%%%%%%%%%%%%%%%%%%%%%%%%%%%%%%%%%%%%%%%%%%%%%%%%%%%%%%
% 定义 \NOTE 命令
%%%%%%%%%%%%%%%%%%%%%%%%%%%%%%%%%%%%%%%%%%%%%%%%%%%%%%%%%%%%%%%%%%%
\newenvironment{inner_note}[1][.88]
{
    \begin{minipage}{#1\textwidth}
        \colorbox{teal}{\textcolor{yellow}{NOTE}}\\% 提示标题顔色:茶绿背景,黄色字体
}
{
    \end{minipage}
}

\newcommand{\NOTE}[2][.88]{
    \begin{center}
        \fbox{
            \begin{inner_note}[#1]
                #2
            \end{inner_note}
        }
    \end{center}
}

%%%%%%%%%%%%%%%%%%%%%%%%%%%%%%%%%%%%%%%%%%%%%%%%%%%%%%%%%%%%%%%%%%%
% 带缩进的列表
%%%%%%%%%%%%%%%%%%%%%%%%%%%%%%%%%%%%%%%%%%%%%%%%%%%%%%%%%%%%%%%%%%%
\newenvironment{ITEMIZE}[1][1em]
{
    \begin{itemize}[itemindent=#1]
}{
    \end{itemize}
}

\newenvironment{ENUMERATE}[1][1em]
{
    \begin{enumerate}[itemindent=#1]
}{
    \end{enumerate}
}

\newenvironment{DESCRIPTION}[1][1em]
{
    \begin{description}[itemindent=#1]
}{
    \end{description}
}

%%%%%%%%%%%%%%%%%%%%%%%%%%%%%%%%%%
%%
%%%%%%%%%%%%%%%%%%%%%%%%%%%%%%
% 链接引用
%%%%%%%%%%%%%%%%%%%%%%%%%%%%%%%%%%%%%%%%%%%%%%%%%%%%
\usepackage[colorlinks, linkcolor=black, anchorcolor=black, citecolor=black]{hyperref}
%%%%%%%%%%%%%%%

%%%%%%%%%%%%%%%%%%%%%%%%%%%%%%%%%%%%%%%%%%%%%%%%%%%%%%%%%%%%%%%%%%%
% 代码引用
%%%%%%%%%%%%%%%%%%%%%%%%%%%%%%%%%%%%%%%%%%%%%%%%%%%%%%%%%%%%%%%%%%%
\usepackage{listings}
\lstdefinestyle{base}{
    tabsize=4,
    frame=single, % 单线边框
    frameround=tttt, % 边框圆角,f指尖角,t指圆角,如fttt指一个尖角,三个圆角
    %
    xleftmargin=2em,xrightmargin=2em,aboveskip=1em, % 代码框边界与页面的边距
    %
    % lineskip=-0.05em, % 行距调整
    gobble=2, % 缩进字符数量
    %
    basicstyle=\small\color{black}\fontspec{Monaco},
    keywordstyle=\color{orange!50!black},
    stringstyle=\color{red},
    commentstyle=\color{green!50!black},
    backgroundcolor=\color{solarized-base3},
    %
    breaklines=true,
    showstringspaces=false,
    %
    % numbers=right,
    % numberstyle=\tiny,
}

\lstdefinestyle{sh}{
    style=base,
    language={sh},
}

\lstdefinestyle{rust}{
    style=base,
    language={[GNU]C++},
    morekeywords={in,let,fn},
}

\lstdefinestyle{c}{
    style=base,
    language={[GNU]C++},
}

\lstdefinestyle{go}{
    style=base,
    language={[GNU]C++},
    morekeywords={go,func},
}

\lstset{
    style=sh,
}

%%%%%%%%%%%%%%%%%%%%%%%%%%%%%%%%%%%%%%%%%%%%%%%%%%%%%%%%%%%%%%%%%%%
% 行内代码引用样式
%%%%%%%%%%%%%%%%%%%%%%%%%%%%%%%%%%%%%%%%%%%%%%%%%%%%%%%%%%%%%%%%%%%
\newcommand{\LSTINLINE}[2][sh]{\colorbox{solarized-base3}{\lstinline[style=#1]{#2}}}


\title{\huge\textbf{《西说FreeBSD》}}
\author{\href{https://github.com/XiMen-classroom}{\LARGE \textbf{西~门}}}
\date{二零一九年五月一日}

%%%%%%%%%%%%%%%%%%%%%%%%%%%%%%%%%
\begin{document}
%%%%%%%%%%%%%%%%%%%%%%%%%%%%%%%%%

\thispagestyle{empty}
\maketitle

~\par
西门?何许人也?

西门,本名范辉,西门课堂创始人,基础平台架构师,Linux专家,一万小时程序员,一线大厂资深后端,区块链拓荒者,Rust语言布道者$\cdots$%
好吧,说人话,十年老码农,技术狂,热爱开源,乐于分享。

~\par
欢迎来到西门课堂,听西门老师说FreeBSD!

~\par
FreeBSD是什么?

FreeBSD是BSD\footnote{BSD,Berkeley Software Distribution,最早的UNIX衍生系统,1977至1995年间由加州大学伯克利分校开发}类操作系统的一个核心分支,在世界范围内都有广泛的应用,至今已有30多年的历史。关于BSD,不得不提的是:
\begin{ITEMIZE}
    \item TCP/IP协议族的基石---sockets\footnotemark,是由BSD发明的
    \item Linux诞生与发展的过程中,大量借鉴了BSD的设计
    \item 苹果公司的手机操作系统IOS是基于FreeBSD开发的
    \item 苹果公司的电脑操作系统Mac OS X是基于FreeBSD开发的
    \item 思科公司的网络设备嵌入式操作系统IOS,是基于FreeBSD开发的
\end{ITEMIZE}
\footnotetext{Berkeley sockets,伯克利套接字,详情参见\href{https://en.wikipedia.org/wiki/Berkeley_sockets}{维基百科}}

~\par
本书风格是提纲式的解说,力求简单明了,需要详细步骤演示的同学,请结合\href{https://space.bilibili.com/393582752}{视频教程}一起学习,西门课堂的B站主页是:

\href{https://space.bilibili.com/393582752}{space.bilibili.com/393582752}

~\par
\begin{center}
    \includegraphics[width=0.3\textwidth]{figs/WeChat.jpg}\\
    \textcolor{gray}{扫码加微信,欢迎技术交流}
\end{center}

% 目录页码使用罗马数字
\pagenumbering{Roman}
\tableofcontents

% 正文使用阿拉伯数字
\clearpage
\pagenumbering{arabic}

\chapter{基础篇}
\clearpage

%%%%%%%%%%%%%%%%%%%%%%%
\section{系统安装}
%%%%%%%%%%%%%%%%%%%%%%%

\subsection{准备虚拟机环境}

可选Qemu、VirtualBox 、Vmware等虚拟方案,挑自己熟悉的即可。本书使用的是Linux平台上的Qemu方案,启动脚本参见第\pageref{ap:qemu}页的附录\ref{ap:qemu}。

\subsection{获取系统盘并安装}

\NOTE{%
    官方提供了现成的多种格式的虚拟机镜像%
    \footnote{不同的虚拟机需要的镜像格式不同,Qemu选择后缀名qcow2,Vmware或VirtualBox选择后缀名vmdk},%
    \href{http://mirrors.ustc.edu.cn/freebsd/releases/VM-IMAGES/12.0-RELEASE/amd64/Latest}{下载}%
    \footnote{下载地址:\href{http://mirrors.ustc.edu.cn/freebsd/releases/VM-IMAGES/12.0-RELEASE/amd64/Latest}{mirrors.ustc.edu.cn/freebsd/releases/VM-IMAGES/12.0-RELEASE/amd64/Latest/}}%
    完成后,将镜像导入虚拟机,就可以跳过安装环节,直接学习后续内容。%
}

国内访问官网比较慢,建议从\href{http://mirrors.ustc.edu.cn/freebsd/releases/ISO-IMAGES/12.0/}{中科大镜像站}下载系统安装盘。

FreeBSD的安装过程非常简洁,按照向导一步步操作即可,初次操作可以全部使用默认选项。如有不解,可登陆西门课堂的B站主页观看\href{https://space.bilibili.com/393582752}{视频教程},或与西门老师微信交流。

%%%%%%%%%%%%%%%%%%%%%%%
\section{用户管理}
%%%%%%%%%%%%%%%%%%%%%%%

\subsection{交互式}

\paragraph{增删用户}~
\begin{lstlisting}
    # 这两个命令会一项一项提示需要填写或选择的内容
    # 最后还会弹出一个结果预览界面,以供确认是否执行
    adduser
    rmuser
\end{lstlisting}

\paragraph{修改用户属性}~

\begin{lstlisting}
    # 如果不指定用户名,默认修改当前用户自己的信息
    # 这个命令会弹出一个文本编辑界面,保存退出后会批量应用更改的内容
    # 默认启动的编辑器是vim
    chpass [username]
\end{lstlisting}

\subsection{命令式}

pw是FreeBSD下用户管理的高级通用工具\footnote{功能相当于Linux下useradd + userdel + usermod三者的综合},提供了各种细化的配置选项,接下来给出几个常用功能的示例。

\paragraph{添加用户或组}~

\begin{lstlisting}
    # -n 指定用户名,-u 指定 UID,-c 指定备注信息,-s 指定登陆 SHELL
    # -w 表示初始密码设定策略,可选值有 yes(密码与用户名相同)、random(密码随机)等
    # -G 将新用户加入 wheel、sshd 两个用户组,-m 表示创建家目录
    pw useradd [-n] zhangsan -u 20000 -c 张三 -s /bin/sh -w random -G wheel,sshd -m

    # -M 指定组成员
    pw groupadd [-n] mygrp -g 20000 -M zhangsan,lisi,wangwu
\end{lstlisting}

\paragraph{查看用户或组信息}~

\begin{lstlisting}
    # -n 选项接受用户名或 UID为参数,用于标识要操作的用户,并且是可以省略的
    # 后续命令的 -n 选项含义,均与此相同
    # -P 以人类宜读的形式打印信息
    # -a 显示所有用户的信息
    pw usershow [-n] zhangsan -P
    pw usershow -a

    # -n 指定组名或 GID
    pw groupshow [-n] wheel -P
    pw groupshow -a
\end{lstlisting}

\paragraph{删除用户或组}~

\begin{lstlisting}
    # -r 表示同时删除用户家目录
    pw userdel zhangsan -r
    pw userdel 20000 -r

    pw groupdel video
\end{lstlisting}

\paragraph{更改用户或组的属性}~

\begin{lstlisting}
    # 除 -n 外,其它选项都是用于指定要更改的内容,含义大多与添加用户时是一样的
    # 比较特别是 -l 选项,用于指定新的用户名
    pw usermod zhangsan -u 30000 -d /tmp/lisi -g sshd -G video,audio -s /bin/csh
    pw usermod 20000 -l lisi -c 李四

    # -g 指定新的 GID,-l 指定新的组名,-d 指定要删除的组成员
    pw groupmod video -g 30000 -l VIDEO -d zhangsan,lisi
\end{lstlisting}

\paragraph{锁定与解锁用户}~

\begin{lstlisting}
    # 锁定之后无法登陆,解锁之后恢复登陆
    pw lock zhangsan
    pw unlock zhangsan
\end{lstlisting}
%%%%%%%%%%%%%%%%%%%%%%%
\section{软件管理}
%%%%%%%%%%%%%%%%%%%%%%%

\subsection{启用国内软件源}

顺序执行如下命令,启用中科大软件源:
\begin{lstlisting}
    # pkg 源
    mkdir -p /usr/local/etc/pkg/repos
    cd !$
    cat<<!>FreeBSD.conf
    FreeBSD: { url: "pkg+http://mirrors.ustc.edu.cn/freebsd-pkg/${ABI}/quarterly" }
    !

    # port 源
    cat<<!>/etc/make.conf
    MASTER_SITE_OVERRIDE?=http://mirrors.ustc.edu.cn/freebsd-ports/distfiles/${DIST_SUBDIR}
    !
\end{lstlisting}

\subsection{pkg方式}
直接下载安装已经编译好的二进制包\footnote{类似于Ubuntu Linux的apt},常用操作如下:

\vspace{1ex}
\begin{minipage}{\textwidth}
    \small
    \begin{tabular}{|l|l|}
        \hline
        pkg install <pkg name>&安装\\
        pkg delete <pkg name>&卸载\\
        pkg autoremove&清理\\
        pkg -N&统计已安装的非系统包数量\\
        pkg info&列出所有已安装的非系统包\\
        pkg info <pkg name>&显示指定包的详细信息\\
        pkg which <file path>&查询指定文件来源于哪个包\\
        pkg check -s -a|<pkg name>&检查包的完整性(checksum)\\
        pkg check -d -a|<pkg name>&检查并自动安装缺失的依赖包\\
        pkg audit&对所有软件包进行安全审计\\
        \hline
    \end{tabular}
\end{minipage}

\subsection{port方式}
从源码编译安装\footnote{类似于Gentoo Linux的emerge},常用操作如下:

\vspace{1ex}
\begin{minipage}{\textwidth}
    \small
    \begin{tabular}{|l|l|}
        \hline
        portsnap fetch&下载最新的源码库快照\\
        portsnap extract&展开快照,仅在首次同步快照时需要\\
        portsnap update\footnote{通常合并执行:portsnap fetch [extract] update}&更新本地仓库快照\\
        make config\footnote{执行任何make命令之前,需要首先切换到目标软件的源码路径}&配置编译选项\\
        make [-jN] install&[N路并行]编译安装\\
        make clean\footnote{通常合并执行:make [-j5] install clean}&清理临时文件\\
        make deinstall&卸载\\
        make reinstall&重装\\
        \hline
    \end{tabular}
\end{minipage}

\subsection{系统自身更新}

\subsubsection{小版本更新}
使用GENERIC内核的情况下,更新系统非常简单,执行\LSTINLINE{freebsd-update fetch install}即可;使用自定义内核的情况下,需要在更新完毕后重新编译内核。

\subsubsection{大版本更新}
生产环境中极少在既有系统之上进行操作系统的大版本更新,此处就不做无谓的论述了,有兴趣的同学可以参看\href{https://www.freebsd.org/doc/en_US.ISO8859-1/books/handbook/updating-upgrading.html}{官方文档}。

%%%%%%%%%%%%%%%%%%%%%%%
\section{服务管理}
%%%%%%%%%%%%%%%%%%%%%%%

\subsection{配置文件}
服务管理的配置文件有三个:

\vspace{1ex}
\begin{tabular}{|l|l|}
    \hline
    /etc/rc.conf.local&优先级最高\\
    /etc/rc.conf&优先级中等\\
    /etc/default/rc.conf&优先级最低\\
    \hline
\end{tabular}

\vspace{1ex}
在其中以\LSTINLINE{<服务名称>_enable="YES"}的格式写入,即表示开机启动某服务。

\subsection{手动启停}

\paragraph{已设置随机启动的服务}~
\begin{lstlisting}
    service <服务名称> start
    service <服务名称> restart
    service <服务名称> stop
\end{lstlisting}

\paragraph{未设置随机启动的服务}~
\begin{lstlisting}
    # 临时服务,即非随机启动的服务
    # 其所有的子命令都需在标准子命令前加 one 前缀
    service <服务名称> onestart
    service <服务名称> onerestart
    service <服务名称> onestop
\end{lstlisting}

\subsection{自定义服务}

为非系统服务设置开机启动,通常有两种方式:
\begin{ITEMIZE}
    \item 在\,/etc/rc.local\,中追加指定服务的启动命令\footnote{已废弃但目前仍然可用,不推荐}
    \item 在\,/usr/local/etc/rc.d\,中放置自定义的服务管理脚本
\end{ITEMIZE}

第一种方式是几乎所有类UNIX系统都支持或曾经支持的传统启动方式,目前在FreeBSD和大多数Linux发行版中都处于``已废弃但仍然可用''的状态;这里重点讲一下第二种方式中提到的服务管理脚本的书写格式,样板示例\footnote{具体到某个服务的实际启动脚本,可以到\,/etc/rc.d/\,路径下查看}如下:

\begin{lstlisting}
    #!/bin/sh

    ######## 注意!!! 以下两行内容不是注释 ########
    # PROVIDE: <服务名称>
    # REQUIRE: <必须在这之前启动的服务列表,逗号分割>

    # 服务名称,如 "sshd"
    name="<服务名称>"

    # 指定用于 /etc/rc.conf.local 等配置文件中的开机启动語法
    # 如:此处 $name 设置为 sshd,则自启语法就是 sshd_enable="YES"
    rcvar=${name}_enable

    # 如: /usr/local/bin/sshd
    command="<可执行文件的路径>"

    # 可选:指定服务的 pid 文件存储路径,方便管理
    # pidfile="<pid 路径>"

    # 除 start/restart/onestart/onrestart/stop 等系统预置的子命令外,
    # 在此列出用户自定义的其它子命令名称,以空格分割
    extra_commands="<自定义子命令-1> <自定义子命令-2>"

    # 用户自定义的子命令的具体实现,有两种方式:
    # - 比较简单的直接在参数中写内容
    # - 相对复杂的在参数中写自定义的函数的名称,并在之后实现该函数
    <自定义子命令-1>_cmd="echo Hello World"
    <自定义子命令-2>_cmd="do_<自定义子命令-2>"

    do_<自定义子命令-2>() {
        echo "Hello World Again"
    }

    # 以下三项为固定格式,用于设置系统预定义的环境
    . /etc/rc.subr
    load_rc_config $name
    run_rc_command "$1"
\end{lstlisting}
%%%%%%%%%%%%%%%%%%%%%%%
\section{网络配置}
%%%%%%%%%%%%%%%%%%%%%%%

\subsection{地址与路由}

\paragraph{手动管理}~

\begin{lstlisting}
    # 查看已建立的网络连接与服务端口
    sockstat -c
    sockstat -4 # IPV4
    sockstat -6 # IPV6

    # 设置 IP
    ifconfig <网卡名称> 192.168.1.99 netmask 255.255.225.0

    # 配置路由
	route show 172.16.10.0                      # 显示指定网络的路由信息
	route add -net 172.16.10.0/24 172.16.1.1    # 为特定网络设定静态路由
	route add -net 0.0.0.0/0 192.168.1.1        # 设置默认路由
	route add default 192.168.1.1               # 设置默认路由,简短语法
	route change -net 172.16.10.0/24 172.16.1.2 # 更改静态路由
	route delete -net 172.16.10.0/24 172.16.1.2 # 删除网络的指定路由
	route flush                                 # 删除本机所有路由信息
\end{lstlisting}

\paragraph{配置文件}~

也可以写在rc.conf.local等配置文件中,具体写法\,man rc.conf(5)\footnote{快速定位至网络配置,在手册中搜索network\_interfaces}。

\subsection{FTP}
ftpd是FreeBSD自带的一个精简实用的ftp服务器\footnote{如果名称为ftp的用户存在,且不在黑名单中,则任意用户可使用ftp匿名登陆服务器,可见范围被限制在ftp的家目录下}。
\begin{lstlisting}
    # 黑名单
    echo "root" >> /etc/ftpusers

    # 将所有用户锁定在指定目录(/home/ftp)下,禁止查看外部目录結构
    echo "@ /home/ftp" >> /etc/ftpchroot

    # 开机自启
    echo "ftpd_enable="YES"" >> /etc/rc.conf
\end{lstlisting}

\subsection{NTP}
nfpd是FreeBSD自带的时间同步服务器。
\begin{lstlisting}
    # /etc/rc.conf
    ntpd_enable="YES"

    # /etc/ntp.conf:用于陈列上游 NTP 服务器地址
    server    ntp1.nl.net
\end{lstlisting}


%%%%%%%%%%%%%%%%%%%%%%%
\section{桌面环境}
%%%%%%%%%%%%%%%%%%%%%%%

FreeBSD总体来说不适合用作通用的桌面环境,其对新硬件的支持速度远落后于Windows、Linux等系统,桌面软件的数量也较少。

但如果需要的仅仅是一个极简的高效开发环境,那么前面所说的缺点,反而会成为优点,因为太多花里胡哨的东西,只会对你造成干扰。

以下是Intel平台安装xfce桌面的简单示例:

\begin{lstlisting}
    # 安装基本环境
    pkg install xorg xfce

    # 确保开机加载声卡与显卡驱动
    cat<<!>>/boot/loader.conf
    snd_hda_load="YES"
    i915kms_load="YES"
    !

    # 不安装窗口管理器,直接使用 startxfce4 启动桌面
    echo ". /usr/local/etc/xdg/xfce4/xinitrc" > ~/.xinitrc

    # 中文字体
    mkdir -p /usr/local/share/fonts/extra_fonts_dir
    cd !$
    cp -R <你的字体文件存储路径>/* ./
    mkfontdir && mkfontscale && fc-cache -fv

    # 安装你需要的应用软件
    pkg install ibus ibus-table firefox-esr cmake vim ...

    # 使用桌面环境的用户需要加入 video 组
    pw groupmod video -m <你的用户名>

    # 从命令行终端启动桌面
    startx
\end{lstlisting}

\chapter{基础篇}
\clearpage

%%%%%%%%%%%%%%%%%%%%%%%
\section{系统安装}
%%%%%%%%%%%%%%%%%%%%%%%

\subsection{准备虚拟机环境}

可选Qemu、VirtualBox 、Vmware等虚拟方案,挑自己熟悉的即可。本书使用的是Linux平台上的Qemu方案,启动脚本参见第\pageref{ap:qemu}页的附录\ref{ap:qemu}。

\subsection{获取系统盘并安装}

\NOTE{%
    官方提供了现成的多种格式的虚拟机镜像%
    \footnote{不同的虚拟机需要的镜像格式不同,Qemu选择后缀名qcow2,Vmware或VirtualBox选择后缀名vmdk},%
    \href{http://mirrors.ustc.edu.cn/freebsd/releases/VM-IMAGES/12.0-RELEASE/amd64/Latest}{下载}%
    \footnote{下载地址:\href{http://mirrors.ustc.edu.cn/freebsd/releases/VM-IMAGES/12.0-RELEASE/amd64/Latest}{mirrors.ustc.edu.cn/freebsd/releases/VM-IMAGES/12.0-RELEASE/amd64/Latest/}}%
    完成后,将镜像导入虚拟机,就可以跳过安装环节,直接学习后续内容。%
}

国内访问官网比较慢,建议从\href{http://mirrors.ustc.edu.cn/freebsd/releases/ISO-IMAGES/12.0/}{中科大镜像站}下载系统安装盘。

FreeBSD的安装过程非常简洁,按照向导一步步操作即可,初次操作可以全部使用默认选项。如有不解,可登陆西门课堂的B站主页观看\href{https://space.bilibili.com/393582752}{视频教程},或与西门老师微信交流。

%%%%%%%%%%%%%%%%%%%%%%%
\section{用户管理}
%%%%%%%%%%%%%%%%%%%%%%%

\subsection{交互式}

\paragraph{增删用户}~
\begin{lstlisting}
    # 这两个命令会一项一项提示需要填写或选择的内容
    # 最后还会弹出一个结果预览界面,以供确认是否执行
    adduser
    rmuser
\end{lstlisting}

\paragraph{修改用户属性}~

\begin{lstlisting}
    # 如果不指定用户名,默认修改当前用户自己的信息
    # 这个命令会弹出一个文本编辑界面,保存退出后会批量应用更改的内容
    # 默认启动的编辑器是vim
    chpass [username]
\end{lstlisting}

\subsection{命令式}

pw是FreeBSD下用户管理的高级通用工具\footnote{功能相当于Linux下useradd + userdel + usermod三者的综合},提供了各种细化的配置选项,接下来给出几个常用功能的示例。

\paragraph{添加用户或组}~

\begin{lstlisting}
    # -n 指定用户名,-u 指定 UID,-c 指定备注信息,-s 指定登陆 SHELL
    # -w 表示初始密码设定策略,可选值有 yes(密码与用户名相同)、random(密码随机)等
    # -G 将新用户加入 wheel、sshd 两个用户组,-m 表示创建家目录
    pw useradd [-n] zhangsan -u 20000 -c 张三 -s /bin/sh -w random -G wheel,sshd -m

    # -M 指定组成员
    pw groupadd [-n] mygrp -g 20000 -M zhangsan,lisi,wangwu
\end{lstlisting}

\paragraph{查看用户或组信息}~

\begin{lstlisting}
    # -n 选项接受用户名或 UID为参数,用于标识要操作的用户,并且是可以省略的
    # 后续命令的 -n 选项含义,均与此相同
    # -P 以人类宜读的形式打印信息
    # -a 显示所有用户的信息
    pw usershow [-n] zhangsan -P
    pw usershow -a

    # -n 指定组名或 GID
    pw groupshow [-n] wheel -P
    pw groupshow -a
\end{lstlisting}

\paragraph{删除用户或组}~

\begin{lstlisting}
    # -r 表示同时删除用户家目录
    pw userdel zhangsan -r
    pw userdel 20000 -r

    pw groupdel video
\end{lstlisting}

\paragraph{更改用户或组的属性}~

\begin{lstlisting}
    # 除 -n 外,其它选项都是用于指定要更改的内容,含义大多与添加用户时是一样的
    # 比较特别是 -l 选项,用于指定新的用户名
    pw usermod zhangsan -u 30000 -d /tmp/lisi -g sshd -G video,audio -s /bin/csh
    pw usermod 20000 -l lisi -c 李四

    # -g 指定新的 GID,-l 指定新的组名,-d 指定要删除的组成员
    pw groupmod video -g 30000 -l VIDEO -d zhangsan,lisi
\end{lstlisting}

\paragraph{锁定与解锁用户}~

\begin{lstlisting}
    # 锁定之后无法登陆,解锁之后恢复登陆
    pw lock zhangsan
    pw unlock zhangsan
\end{lstlisting}
%%%%%%%%%%%%%%%%%%%%%%%
\section{软件管理}
%%%%%%%%%%%%%%%%%%%%%%%

\subsection{启用国内软件源}

顺序执行如下命令,启用中科大软件源:
\begin{lstlisting}
    # pkg 源
    mkdir -p /usr/local/etc/pkg/repos
    cd !$
    cat<<!>FreeBSD.conf
    FreeBSD: { url: "pkg+http://mirrors.ustc.edu.cn/freebsd-pkg/${ABI}/quarterly" }
    !

    # port 源
    cat<<!>/etc/make.conf
    MASTER_SITE_OVERRIDE?=http://mirrors.ustc.edu.cn/freebsd-ports/distfiles/${DIST_SUBDIR}
    !
\end{lstlisting}

\subsection{pkg方式}
直接下载安装已经编译好的二进制包\footnote{类似于Ubuntu Linux的apt},常用操作如下:

\vspace{1ex}
\begin{minipage}{\textwidth}
    \small
    \begin{tabular}{|l|l|}
        \hline
        pkg install <pkg name>&安装\\
        pkg delete <pkg name>&卸载\\
        pkg autoremove&清理\\
        pkg -N&统计已安装的非系统包数量\\
        pkg info&列出所有已安装的非系统包\\
        pkg info <pkg name>&显示指定包的详细信息\\
        pkg which <file path>&查询指定文件来源于哪个包\\
        pkg check -s -a|<pkg name>&检查包的完整性(checksum)\\
        pkg check -d -a|<pkg name>&检查并自动安装缺失的依赖包\\
        pkg audit&对所有软件包进行安全审计\\
        \hline
    \end{tabular}
\end{minipage}

\subsection{port方式}
从源码编译安装\footnote{类似于Gentoo Linux的emerge},常用操作如下:

\vspace{1ex}
\begin{minipage}{\textwidth}
    \small
    \begin{tabular}{|l|l|}
        \hline
        portsnap fetch&下载最新的源码库快照\\
        portsnap extract&展开快照,仅在首次同步快照时需要\\
        portsnap update\footnote{通常合并执行:portsnap fetch [extract] update}&更新本地仓库快照\\
        make config\footnote{执行任何make命令之前,需要首先切换到目标软件的源码路径}&配置编译选项\\
        make [-jN] install&[N路并行]编译安装\\
        make clean\footnote{通常合并执行:make [-j5] install clean}&清理临时文件\\
        make deinstall&卸载\\
        make reinstall&重装\\
        \hline
    \end{tabular}
\end{minipage}

\subsection{系统自身更新}

\subsubsection{小版本更新}
使用GENERIC内核的情况下,更新系统非常简单,执行\LSTINLINE{freebsd-update fetch install}即可;使用自定义内核的情况下,需要在更新完毕后重新编译内核。

\subsubsection{大版本更新}
生产环境中极少在既有系统之上进行操作系统的大版本更新,此处就不做无谓的论述了,有兴趣的同学可以参看\href{https://www.freebsd.org/doc/en_US.ISO8859-1/books/handbook/updating-upgrading.html}{官方文档}。

%%%%%%%%%%%%%%%%%%%%%%%
\section{服务管理}
%%%%%%%%%%%%%%%%%%%%%%%

\subsection{配置文件}
服务管理的配置文件有三个:

\vspace{1ex}
\begin{tabular}{|l|l|}
    \hline
    /etc/rc.conf.local&优先级最高\\
    /etc/rc.conf&优先级中等\\
    /etc/default/rc.conf&优先级最低\\
    \hline
\end{tabular}

\vspace{1ex}
在其中以\LSTINLINE{<服务名称>_enable="YES"}的格式写入,即表示开机启动某服务。

\subsection{手动启停}

\paragraph{已设置随机启动的服务}~
\begin{lstlisting}
    service <服务名称> start
    service <服务名称> restart
    service <服务名称> stop
\end{lstlisting}

\paragraph{未设置随机启动的服务}~
\begin{lstlisting}
    # 临时服务,即非随机启动的服务
    # 其所有的子命令都需在标准子命令前加 one 前缀
    service <服务名称> onestart
    service <服务名称> onerestart
    service <服务名称> onestop
\end{lstlisting}

\subsection{自定义服务}

为非系统服务设置开机启动,通常有两种方式:
\begin{ITEMIZE}
    \item 在\,/etc/rc.local\,中追加指定服务的启动命令\footnote{已废弃但目前仍然可用,不推荐}
    \item 在\,/usr/local/etc/rc.d\,中放置自定义的服务管理脚本
\end{ITEMIZE}

第一种方式是几乎所有类UNIX系统都支持或曾经支持的传统启动方式,目前在FreeBSD和大多数Linux发行版中都处于``已废弃但仍然可用''的状态;这里重点讲一下第二种方式中提到的服务管理脚本的书写格式,样板示例\footnote{具体到某个服务的实际启动脚本,可以到\,/etc/rc.d/\,路径下查看}如下:

\begin{lstlisting}
    #!/bin/sh

    ######## 注意!!! 以下两行内容不是注释 ########
    # PROVIDE: <服务名称>
    # REQUIRE: <必须在这之前启动的服务列表,逗号分割>

    # 服务名称,如 "sshd"
    name="<服务名称>"

    # 指定用于 /etc/rc.conf.local 等配置文件中的开机启动語法
    # 如:此处 $name 设置为 sshd,则自启语法就是 sshd_enable="YES"
    rcvar=${name}_enable

    # 如: /usr/local/bin/sshd
    command="<可执行文件的路径>"

    # 可选:指定服务的 pid 文件存储路径,方便管理
    # pidfile="<pid 路径>"

    # 除 start/restart/onestart/onrestart/stop 等系统预置的子命令外,
    # 在此列出用户自定义的其它子命令名称,以空格分割
    extra_commands="<自定义子命令-1> <自定义子命令-2>"

    # 用户自定义的子命令的具体实现,有两种方式:
    # - 比较简单的直接在参数中写内容
    # - 相对复杂的在参数中写自定义的函数的名称,并在之后实现该函数
    <自定义子命令-1>_cmd="echo Hello World"
    <自定义子命令-2>_cmd="do_<自定义子命令-2>"

    do_<自定义子命令-2>() {
        echo "Hello World Again"
    }

    # 以下三项为固定格式,用于设置系统预定义的环境
    . /etc/rc.subr
    load_rc_config $name
    run_rc_command "$1"
\end{lstlisting}
%%%%%%%%%%%%%%%%%%%%%%%
\section{网络配置}
%%%%%%%%%%%%%%%%%%%%%%%

\subsection{地址与路由}

\paragraph{手动管理}~

\begin{lstlisting}
    # 查看已建立的网络连接与服务端口
    sockstat -c
    sockstat -4 # IPV4
    sockstat -6 # IPV6

    # 设置 IP
    ifconfig <网卡名称> 192.168.1.99 netmask 255.255.225.0

    # 配置路由
	route show 172.16.10.0                      # 显示指定网络的路由信息
	route add -net 172.16.10.0/24 172.16.1.1    # 为特定网络设定静态路由
	route add -net 0.0.0.0/0 192.168.1.1        # 设置默认路由
	route add default 192.168.1.1               # 设置默认路由,简短语法
	route change -net 172.16.10.0/24 172.16.1.2 # 更改静态路由
	route delete -net 172.16.10.0/24 172.16.1.2 # 删除网络的指定路由
	route flush                                 # 删除本机所有路由信息
\end{lstlisting}

\paragraph{配置文件}~

也可以写在rc.conf.local等配置文件中,具体写法\,man rc.conf(5)\footnote{快速定位至网络配置,在手册中搜索network\_interfaces}。

\subsection{FTP}
ftpd是FreeBSD自带的一个精简实用的ftp服务器\footnote{如果名称为ftp的用户存在,且不在黑名单中,则任意用户可使用ftp匿名登陆服务器,可见范围被限制在ftp的家目录下}。
\begin{lstlisting}
    # 黑名单
    echo "root" >> /etc/ftpusers

    # 将所有用户锁定在指定目录(/home/ftp)下,禁止查看外部目录結构
    echo "@ /home/ftp" >> /etc/ftpchroot

    # 开机自启
    echo "ftpd_enable="YES"" >> /etc/rc.conf
\end{lstlisting}

\subsection{NTP}
nfpd是FreeBSD自带的时间同步服务器。
\begin{lstlisting}
    # /etc/rc.conf
    ntpd_enable="YES"

    # /etc/ntp.conf:用于陈列上游 NTP 服务器地址
    server    ntp1.nl.net
\end{lstlisting}


%%%%%%%%%%%%%%%%%%%%%%%
\section{桌面环境}
%%%%%%%%%%%%%%%%%%%%%%%

FreeBSD总体来说不适合用作通用的桌面环境,其对新硬件的支持速度远落后于Windows、Linux等系统,桌面软件的数量也较少。

但如果需要的仅仅是一个极简的高效开发环境,那么前面所说的缺点,反而会成为优点,因为太多花里胡哨的东西,只会对你造成干扰。

以下是Intel平台安装xfce桌面的简单示例:

\begin{lstlisting}
    # 安装基本环境
    pkg install xorg xfce

    # 确保开机加载声卡与显卡驱动
    cat<<!>>/boot/loader.conf
    snd_hda_load="YES"
    i915kms_load="YES"
    !

    # 不安装窗口管理器,直接使用 startxfce4 启动桌面
    echo ". /usr/local/etc/xdg/xfce4/xinitrc" > ~/.xinitrc

    # 中文字体
    mkdir -p /usr/local/share/fonts/extra_fonts_dir
    cd !$
    cp -R <你的字体文件存储路径>/* ./
    mkfontdir && mkfontscale && fc-cache -fv

    # 安装你需要的应用软件
    pkg install ibus ibus-table firefox-esr cmake vim ...

    # 使用桌面环境的用户需要加入 video 组
    pw groupmod video -m <你的用户名>

    # 从命令行终端启动桌面
    startx
\end{lstlisting}

\chapter{基础篇}
\clearpage

%%%%%%%%%%%%%%%%%%%%%%%
\section{系统安装}
%%%%%%%%%%%%%%%%%%%%%%%

\subsection{准备虚拟机环境}

可选Qemu、VirtualBox 、Vmware等虚拟方案,挑自己熟悉的即可。本书使用的是Linux平台上的Qemu方案,启动脚本参见第\pageref{ap:qemu}页的附录\ref{ap:qemu}。

\subsection{获取系统盘并安装}

\NOTE{%
    官方提供了现成的多种格式的虚拟机镜像%
    \footnote{不同的虚拟机需要的镜像格式不同,Qemu选择后缀名qcow2,Vmware或VirtualBox选择后缀名vmdk},%
    \href{http://mirrors.ustc.edu.cn/freebsd/releases/VM-IMAGES/12.0-RELEASE/amd64/Latest}{下载}%
    \footnote{下载地址:\href{http://mirrors.ustc.edu.cn/freebsd/releases/VM-IMAGES/12.0-RELEASE/amd64/Latest}{mirrors.ustc.edu.cn/freebsd/releases/VM-IMAGES/12.0-RELEASE/amd64/Latest/}}%
    完成后,将镜像导入虚拟机,就可以跳过安装环节,直接学习后续内容。%
}

国内访问官网比较慢,建议从\href{http://mirrors.ustc.edu.cn/freebsd/releases/ISO-IMAGES/12.0/}{中科大镜像站}下载系统安装盘。

FreeBSD的安装过程非常简洁,按照向导一步步操作即可,初次操作可以全部使用默认选项。如有不解,可登陆西门课堂的B站主页观看\href{https://space.bilibili.com/393582752}{视频教程},或与西门老师微信交流。

%%%%%%%%%%%%%%%%%%%%%%%
\section{用户管理}
%%%%%%%%%%%%%%%%%%%%%%%

\subsection{交互式}

\paragraph{增删用户}~
\begin{lstlisting}
    # 这两个命令会一项一项提示需要填写或选择的内容
    # 最后还会弹出一个结果预览界面,以供确认是否执行
    adduser
    rmuser
\end{lstlisting}

\paragraph{修改用户属性}~

\begin{lstlisting}
    # 如果不指定用户名,默认修改当前用户自己的信息
    # 这个命令会弹出一个文本编辑界面,保存退出后会批量应用更改的内容
    # 默认启动的编辑器是vim
    chpass [username]
\end{lstlisting}

\subsection{命令式}

pw是FreeBSD下用户管理的高级通用工具\footnote{功能相当于Linux下useradd + userdel + usermod三者的综合},提供了各种细化的配置选项,接下来给出几个常用功能的示例。

\paragraph{添加用户或组}~

\begin{lstlisting}
    # -n 指定用户名,-u 指定 UID,-c 指定备注信息,-s 指定登陆 SHELL
    # -w 表示初始密码设定策略,可选值有 yes(密码与用户名相同)、random(密码随机)等
    # -G 将新用户加入 wheel、sshd 两个用户组,-m 表示创建家目录
    pw useradd [-n] zhangsan -u 20000 -c 张三 -s /bin/sh -w random -G wheel,sshd -m

    # -M 指定组成员
    pw groupadd [-n] mygrp -g 20000 -M zhangsan,lisi,wangwu
\end{lstlisting}

\paragraph{查看用户或组信息}~

\begin{lstlisting}
    # -n 选项接受用户名或 UID为参数,用于标识要操作的用户,并且是可以省略的
    # 后续命令的 -n 选项含义,均与此相同
    # -P 以人类宜读的形式打印信息
    # -a 显示所有用户的信息
    pw usershow [-n] zhangsan -P
    pw usershow -a

    # -n 指定组名或 GID
    pw groupshow [-n] wheel -P
    pw groupshow -a
\end{lstlisting}

\paragraph{删除用户或组}~

\begin{lstlisting}
    # -r 表示同时删除用户家目录
    pw userdel zhangsan -r
    pw userdel 20000 -r

    pw groupdel video
\end{lstlisting}

\paragraph{更改用户或组的属性}~

\begin{lstlisting}
    # 除 -n 外,其它选项都是用于指定要更改的内容,含义大多与添加用户时是一样的
    # 比较特别是 -l 选项,用于指定新的用户名
    pw usermod zhangsan -u 30000 -d /tmp/lisi -g sshd -G video,audio -s /bin/csh
    pw usermod 20000 -l lisi -c 李四

    # -g 指定新的 GID,-l 指定新的组名,-d 指定要删除的组成员
    pw groupmod video -g 30000 -l VIDEO -d zhangsan,lisi
\end{lstlisting}

\paragraph{锁定与解锁用户}~

\begin{lstlisting}
    # 锁定之后无法登陆,解锁之后恢复登陆
    pw lock zhangsan
    pw unlock zhangsan
\end{lstlisting}
%%%%%%%%%%%%%%%%%%%%%%%
\section{软件管理}
%%%%%%%%%%%%%%%%%%%%%%%

\subsection{启用国内软件源}

顺序执行如下命令,启用中科大软件源:
\begin{lstlisting}
    # pkg 源
    mkdir -p /usr/local/etc/pkg/repos
    cd !$
    cat<<!>FreeBSD.conf
    FreeBSD: { url: "pkg+http://mirrors.ustc.edu.cn/freebsd-pkg/${ABI}/quarterly" }
    !

    # port 源
    cat<<!>/etc/make.conf
    MASTER_SITE_OVERRIDE?=http://mirrors.ustc.edu.cn/freebsd-ports/distfiles/${DIST_SUBDIR}
    !
\end{lstlisting}

\subsection{pkg方式}
直接下载安装已经编译好的二进制包\footnote{类似于Ubuntu Linux的apt},常用操作如下:

\vspace{1ex}
\begin{minipage}{\textwidth}
    \small
    \begin{tabular}{|l|l|}
        \hline
        pkg install <pkg name>&安装\\
        pkg delete <pkg name>&卸载\\
        pkg autoremove&清理\\
        pkg -N&统计已安装的非系统包数量\\
        pkg info&列出所有已安装的非系统包\\
        pkg info <pkg name>&显示指定包的详细信息\\
        pkg which <file path>&查询指定文件来源于哪个包\\
        pkg check -s -a|<pkg name>&检查包的完整性(checksum)\\
        pkg check -d -a|<pkg name>&检查并自动安装缺失的依赖包\\
        pkg audit&对所有软件包进行安全审计\\
        \hline
    \end{tabular}
\end{minipage}

\subsection{port方式}
从源码编译安装\footnote{类似于Gentoo Linux的emerge},常用操作如下:

\vspace{1ex}
\begin{minipage}{\textwidth}
    \small
    \begin{tabular}{|l|l|}
        \hline
        portsnap fetch&下载最新的源码库快照\\
        portsnap extract&展开快照,仅在首次同步快照时需要\\
        portsnap update\footnote{通常合并执行:portsnap fetch [extract] update}&更新本地仓库快照\\
        make config\footnote{执行任何make命令之前,需要首先切换到目标软件的源码路径}&配置编译选项\\
        make [-jN] install&[N路并行]编译安装\\
        make clean\footnote{通常合并执行:make [-j5] install clean}&清理临时文件\\
        make deinstall&卸载\\
        make reinstall&重装\\
        \hline
    \end{tabular}
\end{minipage}

\subsection{系统自身更新}

\subsubsection{小版本更新}
使用GENERIC内核的情况下,更新系统非常简单,执行\LSTINLINE{freebsd-update fetch install}即可;使用自定义内核的情况下,需要在更新完毕后重新编译内核。

\subsubsection{大版本更新}
生产环境中极少在既有系统之上进行操作系统的大版本更新,此处就不做无谓的论述了,有兴趣的同学可以参看\href{https://www.freebsd.org/doc/en_US.ISO8859-1/books/handbook/updating-upgrading.html}{官方文档}。

%%%%%%%%%%%%%%%%%%%%%%%
\section{服务管理}
%%%%%%%%%%%%%%%%%%%%%%%

\subsection{配置文件}
服务管理的配置文件有三个:

\vspace{1ex}
\begin{tabular}{|l|l|}
    \hline
    /etc/rc.conf.local&优先级最高\\
    /etc/rc.conf&优先级中等\\
    /etc/default/rc.conf&优先级最低\\
    \hline
\end{tabular}

\vspace{1ex}
在其中以\LSTINLINE{<服务名称>_enable="YES"}的格式写入,即表示开机启动某服务。

\subsection{手动启停}

\paragraph{已设置随机启动的服务}~
\begin{lstlisting}
    service <服务名称> start
    service <服务名称> restart
    service <服务名称> stop
\end{lstlisting}

\paragraph{未设置随机启动的服务}~
\begin{lstlisting}
    # 临时服务,即非随机启动的服务
    # 其所有的子命令都需在标准子命令前加 one 前缀
    service <服务名称> onestart
    service <服务名称> onerestart
    service <服务名称> onestop
\end{lstlisting}

\subsection{自定义服务}

为非系统服务设置开机启动,通常有两种方式:
\begin{ITEMIZE}
    \item 在\,/etc/rc.local\,中追加指定服务的启动命令\footnote{已废弃但目前仍然可用,不推荐}
    \item 在\,/usr/local/etc/rc.d\,中放置自定义的服务管理脚本
\end{ITEMIZE}

第一种方式是几乎所有类UNIX系统都支持或曾经支持的传统启动方式,目前在FreeBSD和大多数Linux发行版中都处于``已废弃但仍然可用''的状态;这里重点讲一下第二种方式中提到的服务管理脚本的书写格式,样板示例\footnote{具体到某个服务的实际启动脚本,可以到\,/etc/rc.d/\,路径下查看}如下:

\begin{lstlisting}
    #!/bin/sh

    ######## 注意!!! 以下两行内容不是注释 ########
    # PROVIDE: <服务名称>
    # REQUIRE: <必须在这之前启动的服务列表,逗号分割>

    # 服务名称,如 "sshd"
    name="<服务名称>"

    # 指定用于 /etc/rc.conf.local 等配置文件中的开机启动語法
    # 如:此处 $name 设置为 sshd,则自启语法就是 sshd_enable="YES"
    rcvar=${name}_enable

    # 如: /usr/local/bin/sshd
    command="<可执行文件的路径>"

    # 可选:指定服务的 pid 文件存储路径,方便管理
    # pidfile="<pid 路径>"

    # 除 start/restart/onestart/onrestart/stop 等系统预置的子命令外,
    # 在此列出用户自定义的其它子命令名称,以空格分割
    extra_commands="<自定义子命令-1> <自定义子命令-2>"

    # 用户自定义的子命令的具体实现,有两种方式:
    # - 比较简单的直接在参数中写内容
    # - 相对复杂的在参数中写自定义的函数的名称,并在之后实现该函数
    <自定义子命令-1>_cmd="echo Hello World"
    <自定义子命令-2>_cmd="do_<自定义子命令-2>"

    do_<自定义子命令-2>() {
        echo "Hello World Again"
    }

    # 以下三项为固定格式,用于设置系统预定义的环境
    . /etc/rc.subr
    load_rc_config $name
    run_rc_command "$1"
\end{lstlisting}
%%%%%%%%%%%%%%%%%%%%%%%
\section{网络配置}
%%%%%%%%%%%%%%%%%%%%%%%

\subsection{地址与路由}

\paragraph{手动管理}~

\begin{lstlisting}
    # 查看已建立的网络连接与服务端口
    sockstat -c
    sockstat -4 # IPV4
    sockstat -6 # IPV6

    # 设置 IP
    ifconfig <网卡名称> 192.168.1.99 netmask 255.255.225.0

    # 配置路由
	route show 172.16.10.0                      # 显示指定网络的路由信息
	route add -net 172.16.10.0/24 172.16.1.1    # 为特定网络设定静态路由
	route add -net 0.0.0.0/0 192.168.1.1        # 设置默认路由
	route add default 192.168.1.1               # 设置默认路由,简短语法
	route change -net 172.16.10.0/24 172.16.1.2 # 更改静态路由
	route delete -net 172.16.10.0/24 172.16.1.2 # 删除网络的指定路由
	route flush                                 # 删除本机所有路由信息
\end{lstlisting}

\paragraph{配置文件}~

也可以写在rc.conf.local等配置文件中,具体写法\,man rc.conf(5)\footnote{快速定位至网络配置,在手册中搜索network\_interfaces}。

\subsection{FTP}
ftpd是FreeBSD自带的一个精简实用的ftp服务器\footnote{如果名称为ftp的用户存在,且不在黑名单中,则任意用户可使用ftp匿名登陆服务器,可见范围被限制在ftp的家目录下}。
\begin{lstlisting}
    # 黑名单
    echo "root" >> /etc/ftpusers

    # 将所有用户锁定在指定目录(/home/ftp)下,禁止查看外部目录結构
    echo "@ /home/ftp" >> /etc/ftpchroot

    # 开机自启
    echo "ftpd_enable="YES"" >> /etc/rc.conf
\end{lstlisting}

\subsection{NTP}
nfpd是FreeBSD自带的时间同步服务器。
\begin{lstlisting}
    # /etc/rc.conf
    ntpd_enable="YES"

    # /etc/ntp.conf:用于陈列上游 NTP 服务器地址
    server    ntp1.nl.net
\end{lstlisting}


%%%%%%%%%%%%%%%%%%%%%%%
\section{桌面环境}
%%%%%%%%%%%%%%%%%%%%%%%

FreeBSD总体来说不适合用作通用的桌面环境,其对新硬件的支持速度远落后于Windows、Linux等系统,桌面软件的数量也较少。

但如果需要的仅仅是一个极简的高效开发环境,那么前面所说的缺点,反而会成为优点,因为太多花里胡哨的东西,只会对你造成干扰。

以下是Intel平台安装xfce桌面的简单示例:

\begin{lstlisting}
    # 安装基本环境
    pkg install xorg xfce

    # 确保开机加载声卡与显卡驱动
    cat<<!>>/boot/loader.conf
    snd_hda_load="YES"
    i915kms_load="YES"
    !

    # 不安装窗口管理器,直接使用 startxfce4 启动桌面
    echo ". /usr/local/etc/xdg/xfce4/xinitrc" > ~/.xinitrc

    # 中文字体
    mkdir -p /usr/local/share/fonts/extra_fonts_dir
    cd !$
    cp -R <你的字体文件存储路径>/* ./
    mkfontdir && mkfontscale && fc-cache -fv

    # 安装你需要的应用软件
    pkg install ibus ibus-table firefox-esr cmake vim ...

    # 使用桌面环境的用户需要加入 video 组
    pw groupmod video -m <你的用户名>

    # 从命令行终端启动桌面
    startx
\end{lstlisting}


%%%%%%%%%%%%%%%%%%%%%%%%%%%%%%%%%
\end{document}
%%%%%%%%%%%%%%%%%%%%%%%%%%%%%%%%%
