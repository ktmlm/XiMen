
%%%%%%%%%%%%%%%%%%%%%%%
\section{桌面环境}
%%%%%%%%%%%%%%%%%%%%%%%

FreeBSD总体来说不适合用作通用的桌面环境,其对新硬件的支持速度远落后于Windows、Linux等系统,桌面软件的数量也较少。

但如果需要的仅仅是一个极简的高效开发环境,那么前面所说的缺点,反而会成为优点,因为太多花里胡哨的东西,只会对你造成干扰。

以下是Intel平台安装xfce桌面的简单示例:

\begin{lstlisting}
    # 安装基本环境
    pkg install xorg xfce

    # 确保开机加载声卡与显卡驱动
    cat<<!>>/boot/loader.conf
    snd_hda_load="YES"
    i915kms_load="YES"
    !

    # 不安装窗口管理器,直接使用 startxfce4 启动桌面
    echo ". /usr/local/etc/xdg/xfce4/xinitrc" > ~/.xinitrc

    # 中文字体
    mkdir -p /usr/local/share/fonts/extra_fonts_dir
    cd !$
    cp -R <你的字体文件存储路径>/* ./
    mkfontdir && mkfontscale && fc-cache -fv

    # 安装你需要的应用软件
    pkg install ibus ibus-table firefox-esr cmake vim ...

    # 使用桌面环境的用户需要加入 video 组
    pw groupmod video -m <你的用户名>

    # 从命令行终端启动桌面
    startx
\end{lstlisting}