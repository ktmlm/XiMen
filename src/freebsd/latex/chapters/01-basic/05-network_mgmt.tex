%%%%%%%%%%%%%%%%%%%%%%%
\section{网络配置}
%%%%%%%%%%%%%%%%%%%%%%%

\subsection{地址与路由}

\paragraph{手动管理}~

\begin{lstlisting}
    # 查看已建立的网络连接与服务端口
    sockstat -c
    sockstat -4 # IPV4
    sockstat -6 # IPV6

    # 设置 IP
    ifconfig <网卡名称> 192.168.1.99 netmask 255.255.225.0

    # 配置路由
	route show 172.16.10.0                      # 显示指定网络的路由信息
	route add -net 172.16.10.0/24 172.16.1.1    # 为特定网络设定静态路由
	route add -net 0.0.0.0/0 192.168.1.1        # 设置默认路由
	route add default 192.168.1.1               # 设置默认路由,简短语法
	route change -net 172.16.10.0/24 172.16.1.2 # 更改静态路由
	route delete -net 172.16.10.0/24 172.16.1.2 # 删除网络的指定路由
	route flush                                 # 删除本机所有路由信息
\end{lstlisting}

\paragraph{配置文件}~

也可以写在rc.conf.local等配置文件中,具体写法\,man rc.conf(5)\footnote{快速定位至网络配置,在手册中搜索network\_interfaces}。

\subsection{FTP}
ftpd是FreeBSD自带的一个精简实用的ftp服务器\footnote{如果名称为ftp的用户存在,且不在黑名单中,则任意用户可使用ftp匿名登陆服务器,可见范围被限制在ftp的家目录下}。
\begin{lstlisting}
    # 黑名单
    echo "root" >> /etc/ftpusers

    # 将所有用户锁定在指定目录(/home/ftp)下,禁止查看外部目录結构
    echo "@ /home/ftp" >> /etc/ftpchroot

    # 开机自启
    echo "ftpd_enable="YES"" >> /etc/rc.conf
\end{lstlisting}

\subsection{NTP}
nfpd是FreeBSD自带的时间同步服务器。
\begin{lstlisting}
    # /etc/rc.conf
    ntpd_enable="YES"

    # /etc/ntp.conf:用于陈列上游 NTP 服务器地址
    server    ntp1.nl.net
\end{lstlisting}
