%%%%%%%%%%%%%%%%%%%%%%%
\section{服务管理}
%%%%%%%%%%%%%%%%%%%%%%%

\subsection{配置文件}
服务管理的配置文件有三个:

\vspace{1ex}
\begin{tabular}{|l|l|}
    \hline
    /etc/rc.conf.local&优先级最高\\
    /etc/rc.conf&优先级中等\\
    /etc/default/rc.conf&优先级最低\\
    \hline
\end{tabular}

\vspace{1ex}
在其中以\LSTINLINE{<服务名称>_enable="YES"}的格式写入,即表示开机启动某服务。

\subsection{手动启停}

\paragraph{已设置随机启动的服务}~
\begin{lstlisting}
    service <服务名称> start
    service <服务名称> restart
    service <服务名称> stop
\end{lstlisting}

\paragraph{未设置随机启动的服务}~
\begin{lstlisting}
    # 临时服务,即非随机启动的服务
    # 其所有的子命令都需在标准子命令前加 one 前缀
    service <服务名称> onestart
    service <服务名称> onerestart
    service <服务名称> onestop
\end{lstlisting}

\subsection{自定义服务}

为非系统服务设置开机启动,通常有两种方式:
\begin{ITEMIZE}
    \item 在\,/etc/rc.local\,中追加指定服务的启动命令\footnote{已废弃但目前仍然可用,不推荐}
    \item 在\,/usr/local/etc/rc.d\,中放置自定义的服务管理脚本
\end{ITEMIZE}

第一种方式是几乎所有类UNIX系统都支持或曾经支持的传统启动方式,目前在FreeBSD和大多数Linux发行版中都处于``已废弃但仍然可用''的状态;这里重点讲一下第二种方式中提到的服务管理脚本的书写格式,样板示例\footnote{具体到某个服务的实际启动脚本,可以到\,/etc/rc.d/\,路径下查看}如下:

\begin{lstlisting}
    #!/bin/sh

    ######## 注意!!! 以下两行内容不是注释 ########
    # PROVIDE: <服务名称>
    # REQUIRE: <必须在这之前启动的服务列表,逗号分割>

    # 服务名称,如 "sshd"
    name="<服务名称>"

    # 指定用于 /etc/rc.conf.local 等配置文件中的开机启动語法
    # 如:此处 $name 设置为 sshd,则自启语法就是 sshd_enable="YES"
    rcvar=${name}_enable

    # 如: /usr/local/bin/sshd
    command="<可执行文件的路径>"

    # 可选:指定服务的 pid 文件存储路径,方便管理
    # pidfile="<pid 路径>"

    # 除 start/restart/onestart/onrestart/stop 等系统预置的子命令外,
    # 在此列出用户自定义的其它子命令名称,以空格分割
    extra_commands="<自定义子命令-1> <自定义子命令-2>"

    # 用户自定义的子命令的具体实现,有两种方式:
    # - 比较简单的直接在参数中写内容
    # - 相对复杂的在参数中写自定义的函数的名称,并在之后实现该函数
    <自定义子命令-1>_cmd="echo Hello World"
    <自定义子命令-2>_cmd="do_<自定义子命令-2>"

    do_<自定义子命令-2>() {
        echo "Hello World Again"
    }

    # 以下三项为固定格式,用于设置系统预定义的环境
    . /etc/rc.subr
    load_rc_config $name
    run_rc_command "$1"
\end{lstlisting}