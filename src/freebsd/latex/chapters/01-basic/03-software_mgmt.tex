%%%%%%%%%%%%%%%%%%%%%%%
\section{软件管理}
%%%%%%%%%%%%%%%%%%%%%%%

\subsection{启用国内软件源}

顺序执行如下命令,启用中科大软件源:
\begin{lstlisting}
    # pkg 源
    mkdir -p /usr/local/etc/pkg/repos
    cd !$
    cat<<!>FreeBSD.conf
    FreeBSD: { url: "pkg+http://mirrors.ustc.edu.cn/freebsd-pkg/${ABI}/quarterly" }
    !

    # port 源
    cat<<!>/etc/make.conf
    MASTER_SITE_OVERRIDE?=http://mirrors.ustc.edu.cn/freebsd-ports/distfiles/${DIST_SUBDIR}
    !
\end{lstlisting}

\subsection{pkg方式}
直接下载安装已经编译好的二进制包\footnote{类似于Ubuntu Linux的apt},常用操作如下:

\vspace{1ex}
\begin{minipage}{\textwidth}
    \small
    \begin{tabular}{|l|l|}
        \hline
        pkg install <pkg name>&安装\\
        pkg delete <pkg name>&卸载\\
        pkg autoremove&清理\\
        pkg -N&统计已安装的非系统包数量\\
        pkg info&列出所有已安装的非系统包\\
        pkg info <pkg name>&显示指定包的详细信息\\
        pkg which <file path>&查询指定文件来源于哪个包\\
        pkg check -s -a|<pkg name>&检查包的完整性(checksum)\\
        pkg check -d -a|<pkg name>&检查并自动安装缺失的依赖包\\
        pkg audit&对所有软件包进行安全审计\\
        \hline
    \end{tabular}
\end{minipage}

\subsection{port方式}
从源码编译安装\footnote{类似于Gentoo Linux的emerge},常用操作如下:

\vspace{1ex}
\begin{minipage}{\textwidth}
    \small
    \begin{tabular}{|l|l|}
        \hline
        portsnap fetch&下载最新的源码库快照\\
        portsnap extract&展开快照,仅在首次同步快照时需要\\
        portsnap update\footnote{通常合并执行:portsnap fetch [extract] update}&更新本地仓库快照\\
        make config\footnote{执行任何make命令之前,需要首先切换到目标软件的源码路径}&配置编译选项\\
        make [-jN] install&[N路并行]编译安装\\
        make clean\footnote{通常合并执行:make [-j5] install clean}&清理临时文件\\
        make deinstall&卸载\\
        make reinstall&重装\\
        \hline
    \end{tabular}
\end{minipage}

\subsection{系统自身更新}

\subsubsection{小版本更新}
使用GENERIC内核的情况下,更新系统非常简单,执行\LSTINLINE{freebsd-update fetch install}即可;使用自定义内核的情况下,需要在更新完毕后重新编译内核。

\subsubsection{大版本更新}
生产环境中极少在既有系统之上进行操作系统的大版本更新,此处就不做无谓的论述了,有兴趣的同学可以参看\href{https://www.freebsd.org/doc/en_US.ISO8859-1/books/handbook/updating-upgrading.html}{官方文档}。
