%%%%%%%%%%%%%%%%%%%%%%%
\section{用户管理}
%%%%%%%%%%%%%%%%%%%%%%%

\subsection{交互式}

\paragraph{增删用户}~
\begin{lstlisting}
    # 这两个命令会一项一项提示需要填写或选择的内容
    # 最后还会弹出一个结果预览界面,以供确认是否执行
    adduser
    rmuser
\end{lstlisting}

\paragraph{修改用户属性}~

\begin{lstlisting}
    # 如果不指定用户名,默认修改当前用户自己的信息
    # 这个命令会弹出一个文本编辑界面,保存退出后会批量应用更改的内容
    # 默认启动的编辑器是vim
    chpass [username]
\end{lstlisting}

\subsection{命令式}

pw是FreeBSD下用户管理的高级通用工具\footnote{功能相当于Linux下useradd + userdel + usermod三者的综合},提供了各种细化的配置选项,接下来给出几个常用功能的示例。

\paragraph{添加用户或组}~

\begin{lstlisting}
    # -n 指定用户名,-u 指定 UID,-c 指定备注信息,-s 指定登陆 SHELL
    # -w 表示初始密码设定策略,可选值有 yes(密码与用户名相同)、random(密码随机)等
    # -G 将新用户加入 wheel、sshd 两个用户组,-m 表示创建家目录
    pw useradd [-n] zhangsan -u 20000 -c 张三 -s /bin/sh -w random -G wheel,sshd -m

    # -M 指定组成员
    pw groupadd [-n] mygrp -g 20000 -M zhangsan,lisi,wangwu
\end{lstlisting}

\paragraph{查看用户或组信息}~

\begin{lstlisting}
    # -n 选项接受用户名或 UID为参数,用于标识要操作的用户,并且是可以省略的
    # 后续命令的 -n 选项含义,均与此相同
    # -P 以人类宜读的形式打印信息
    # -a 显示所有用户的信息
    pw usershow [-n] zhangsan -P
    pw usershow -a

    # -n 指定组名或 GID
    pw groupshow [-n] wheel -P
    pw groupshow -a
\end{lstlisting}

\paragraph{删除用户或组}~

\begin{lstlisting}
    # -r 表示同时删除用户家目录
    pw userdel zhangsan -r
    pw userdel 20000 -r

    pw groupdel video
\end{lstlisting}

\paragraph{更改用户或组的属性}~

\begin{lstlisting}
    # 除 -n 外,其它选项都是用于指定要更改的内容,含义大多与添加用户时是一样的
    # 比较特别是 -l 选项,用于指定新的用户名
    pw usermod zhangsan -u 30000 -d /tmp/lisi -g sshd -G video,audio -s /bin/csh
    pw usermod 20000 -l lisi -c 李四

    # -g 指定新的 GID,-l 指定新的组名,-d 指定要删除的组成员
    pw groupmod video -g 30000 -l VIDEO -d zhangsan,lisi
\end{lstlisting}

\paragraph{锁定与解锁用户}~

\begin{lstlisting}
    # 锁定之后无法登陆,解锁之后恢复登陆
    pw lock zhangsan
    pw unlock zhangsan
\end{lstlisting}