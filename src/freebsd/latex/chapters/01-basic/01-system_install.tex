%%%%%%%%%%%%%%%%%%%%%%%
\section{系统安装}
%%%%%%%%%%%%%%%%%%%%%%%

\subsection{准备虚拟机环境}

可选Qemu、VirtualBox 、Vmware等虚拟方案,挑自己熟悉的即可。本书使用的是Linux平台上的Qemu方案,启动脚本参见第\pageref{ap:qemu}页的附录\ref{ap:qemu}。

\subsection{获取系统盘并安装}

\NOTE{%
    官方提供了现成的多种格式的虚拟机镜像%
    \footnote{不同的虚拟机需要的镜像格式不同,Qemu选择后缀名qcow2,Vmware或VirtualBox选择后缀名vmdk},%
    \href{http://mirrors.ustc.edu.cn/freebsd/releases/VM-IMAGES/12.0-RELEASE/amd64/Latest}{下载}%
    \footnote{下载地址:\href{http://mirrors.ustc.edu.cn/freebsd/releases/VM-IMAGES/12.0-RELEASE/amd64/Latest}{mirrors.ustc.edu.cn/freebsd/releases/VM-IMAGES/12.0-RELEASE/amd64/Latest/}}%
    完成后,将镜像导入虚拟机,就可以跳过安装环节,直接学习后续内容。%
}

国内访问官网比较慢,建议从\href{http://mirrors.ustc.edu.cn/freebsd/releases/ISO-IMAGES/12.0/}{中科大镜像站}下载系统安装盘。

FreeBSD的安装过程非常简洁,按照向导一步步操作即可,初次操作可以全部使用默认选项。如有不解,可登陆西门课堂的B站主页观看\href{https://space.bilibili.com/393582752}{视频教程},或与西门老师微信交流。
