\section{Qemu/KVM启动脚本}\label{ap:qemu}

\begin{lstlisting}
    #!/usr/bin/env bash
    ifname="enp5s0"             # 宿主机端口名称
    br_name="br0"               # 虚拟网桥名称
    br_addr="192.168.1.100/24"  # 给虚拟网桥配置的 IP
    br_rtaddr="192.168.1.1"     # 虚拟网桥的网关 IP

    iso_path="/tmp/FreeBSD.iso" # 安装盘存放路径
    vm_path="/tmp/qemu"         # 虚拟机路径
    vm_disk="disk.qcow2"        # 虚拟硬盘文件名

    # 配置虚拟网桥
    ip link del $br_name 2>/dev/null
    ip link add $br_name type bridge || exit 1
    ip link set $br_name up || exit 1
    ip addr add $br_addr dev $br_name || exit 1
    ip route replace default via $br_rtaddr dev $br_name || exit 1
    ip addr flush dev $ifname || exit 1
    ip route flush dev $ifname || exit 1
    ip link set $ifname master $br_name || exit 1

    # 按需创建虚拟机路径并进入
    mkdir -p $vm_path || exit 1
    cd $vm_path || exit 1

    # tap 形式联网需要一个回调脚本
    echo "#!/usr/bin/env bash
          ip link set \$1 up && sleep 0.1s;
          ip link set \$1 master $br_name" > tap.sh
    chmod +x tap.sh

    # 按需创建虚拟硬盘
    if [[ 0 -eq `find . -name $vm_disk -type f | wc -l` ]]; then
        qemu-img create -f qcow2 -o size=20G disk.qcow2 || exit 1
    fi

    # 启动虚拟机,参数含义 man qemu-system-x86_64(1)
    qemu-system-x86_64 -smbios type=0,uefi=on -enable-kvm \
        -machine q35,accel=kvm -device intel-iommu \
        -cpu host -smp 4,sockets=4,cores=1,threads=1 \
        -m 4096 \
        -netdev tap,ifname=tap0,script=tap.sh,id=vmNic -device virtio-net-pci,netdev=vmNic \
        -drive file=${vm_path}/${vm_disk},if=none,cache=writeback,id=vmDisk -device virtio-blk-pci,drive=vmDisk \
        -drive file=${iso_path},readonly=on,media=cdrom \
        -boot order=cd \
        -name vmFreeBSD \
        -display curses
\end{lstlisting}
